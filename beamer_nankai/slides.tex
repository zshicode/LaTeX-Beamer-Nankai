\documentclass[hyperref,UTF8,11pt]{beamer}
\usepackage{ctex}
\usepackage[utf8]{inputenc}
\usepackage{fontspec}
\usepackage{comment}
%\setCJKfamilyfont{SimHei}
\usepackage{xeCJK}
%\renewcommand{\CJKfamilydefault}{\CJKsfdefault}
%\setmonofont{Consolas}
\setsansfont{Microsoft YaHei}
\setCJKmainfont{Microsoft YaHei}
%\setCJKmonofont{KaiTi}
%\setCJKsansfont{Microsoft YaHei}
\usefonttheme{professionalfonts}
\usepackage{hyperref}
\usepackage{graphicx}
\graphicspath{{image/}} % storage figure in a sub-folder
% \usepackage[parfill]{parskip} % Activate to begin paragraphs with an empty line rather than an indent
%\usepackage{epstopdf}
\usepackage{bm}
\usepackage{nkcolor}
%\usepackage{hyperref}
\hypersetup{CJKbookmarks=true}
\usepackage{url}
\usepackage{amsmath}
\usepackage{amsthm}
%\theoremstyle{definition}
%\newtheorem{theorem}{定理}
%\newtheorem{definition}{定义}
%\newtheorem{corollary}{推论}
%\newtheorem{example}{例}
\usepackage{booktabs} % for much better looking tables
%\usepackage{cite} % reference
\usepackage[backend=biber,style=numeric,sorting=none]{biblatex}
%\usepackage[backend=biber,style=apalike]{biblatex}
%\usepackage[backend=biber,style=authoryear]{biblatex}
% if style=apalike or authoryear, use \parencite instead of \cite
\addbibresource{nkthesis.bib}
\beamertemplatetextbibitems
\usepackage{array} % for better arrays (eg matrices) in maths
%\usepackage{paralist} % very flexible & customisable lists (eg. enumerate/itemize, etc.)
\usepackage{verbatim} % adds environment for commenting out blocks of text & for better verbatim
\usepackage{subfigure} % make it possible to include more than one captioned figure/table in a single float
% These packages are all incorporated in the memoir class to one degree or another...
%\usepackage{threeparttable}
\usepackage{cases} %equation set
\usepackage{multirow} %use table
\usepackage{enumerate}
\usepackage{algorithm}
\usepackage{algorithmic}
\usepackage{xcolor}
%\usepackage{capt-of}
\setcounter{tocdepth}{1}%只显示section,不显示subsection
\usepackage{listings}
\lstset{tabsize=4, keepspaces=true,
    xleftmargin=2em,xrightmargin=0em, aboveskip=1em,
    backgroundcolor=\color{gray!20},  % 定义背景颜色
    frame=none,                       % 表示不要边框
    extendedchars=false,              % 解决代码跨页时,章节标题,页眉等汉字不显示的问题
    numberstyle=\ttfamily,
    basicstyle=\ttfamily,
    keywordstyle=\color{blue}\bfseries,
    breakindent=10pt,
    identifierstyle=,                 % nothing happens
    commentstyle=\color{green}\small,  % 注释的设置
    morecomment=[l][\color{green}]{\#},
    numbers=left,stepnumber=1,numberstyle=\scriptsize,
    showstringspaces=false,
    showspaces=false,
    flexiblecolumns=true,
    breaklines=true, breakautoindent=true,breakindent=4em,
    escapeinside={/*@}{@*/},
}

\title[Beamer模板]{中文题目}
\subtitle{English Title}
\author[作者]{答辩人:\quad \\ 学号:\quad  \\ 专业:\quad \\ 指导教师:\quad \\\quad}
\institute[学院]{南开大学 \quad 学院}
\date{2019年5月} %Activate to display a given date or no date (if empty),
% otherwise the current date is printed

\begin{document}
%%%%%%%%%% 定理类环境的定义 %%%%%%%%%%
%% 必须在导入中文环境之后
\newcommand{\redstress}[1]{{\color{red}{#1}}}
%\renewcommand{\raggedright}{\leftskip=0pt \rightskip=0pt plus 0cm}
\renewcommand{\contentsname}{目录}     % 将Contents改为目录
\renewcommand{\abstractname}{摘要}     % 将Abstract改为摘要
\renewcommand{\refname}{参考文献}      % 将References改为参考文献
\renewcommand{\indexname}{索引}
\renewcommand{\figurename}{图}
\renewcommand{\tablename}{表}
\renewcommand{\appendixname}{附录}
%\renewcommand{\proofname}{证明}
%\renewcommand{\algorithm}{算法}
%----------------------------------------------------------------------
% Title frame
\begin{frame}
\maketitle
\end{frame}

\section{绪论}

\begin{frame}{Frametitle}
    \begin{itemize}
        \item First
        \item Second
        \item ...
    \end{itemize}
\end{frame}

\begin{frame}{Block}
    \begin{block}{Part 1}
        Test.
    \end{block}
    \begin{theorem}[Thm 1]
        Thm.
    \end{theorem}
    \begin{proof}
        Bingo.
    \end{proof}
\end{frame}

\begin{frame}{Enumerate}
    \begin{equation}
        F=ma\label{eq1}
    \end{equation}
    \begin{enumerate}
        \item First \redstress{important\cite{em1977}}
        \item Second \eqref{eq1}
    \end{enumerate}
\end{frame}

\begin{comment}

This is comment.

\end{comment}

\section{算法}

\begin{frame}{算法}
    \begin{algorithm}[H]
        \caption{算法1}\label{alg:em}
        \begin{algorithmic}[1]
            \REQUIRE Param
            \ENSURE $a$
            \REPEAT
            \STATE Compute $a_n$
            \UNTIL convergence
            \RETURN $a\leftarrow a_n$
        \end{algorithmic}
    \end{algorithm}    
\end{frame}

\begin{frame}{图片}
    \begin{figure}
        \centering
        \includegraphics[width=0.85\textwidth]{nk_logo.png}
        \caption{南开大学}\label{fig:nk}
    \end{figure}
\end{frame}

\begin{frame}{分栏}
    \begin{columns}
        \begin{column}{0.3\textwidth}
            \begin{figure}
                \centering
                \includegraphics[width=0.95\textwidth]{nk_logo.png}
                \caption{NKU}\label{fig:nku}
            \end{figure}
        \end{column}
        \begin{column}{0.7\textwidth}
            \begin{itemize}
                \item ...
            \end{itemize}
        \end{column}
    \end{columns}
\end{frame}

\begin{frame}{Subfigure}
    \begin{figure}
        \centering
        \subfigure[]{\includegraphics[width=0.47\textwidth]{nk_logo.png}}
        \subfigure[]{\includegraphics[width=0.47\textwidth]{nkbackground.png}}
        \caption{Subfigure\footnote{See: \url{www.nankai.edu.cn}}}\label{fig:rnn}
    \end{figure}
\end{frame}

% To put the content of a frame in several pages, use allowframebreaks
\begin{frame}[allowframebreaks]
    \frametitle{Longframe}
    \begin{itemize}
        \item ...
    \end{itemize}
\end{frame}

\section{仿真}

\begin{frame}{More block}
    \begin{exampleblock}{Example}
    Eg1.
    \end{exampleblock}
    \begin{alertblock}{Attention}
        Test block!
    \end{alertblock}
\end{frame}

\begin{frame}{表格}
    \begin{table}[]
        \centering
        \caption{数据}
        \label{tab1}
        \begin{tabular}{@{}ccccc@{}}
        \toprule
                    & $q$         & $r$         & $a$ & $p$           \\ \midrule
        实际值         & $1$  & $5$  & 2   & 3  \\
        方法1          & $4$ & $3$ & 1 & 1\\
        方法2        & $4$ & $3$ & 2 & 2\\
        方法3 & $5$ & $2$ & 3 & 3\\
        方法4          & $4$ & $2$ & 2 & 2\\ \bottomrule
        \end{tabular}
    \end{table}
\end{frame}

\begin{frame}[fragile]
    \frametitle{代码}
    \begin{lstlisting}[language=java]
public class Hello{
    public static void main(String args[]){
        System.out.println("hello,world");
    }
}\end{lstlisting}
\end{frame}

\section{总结与展望}

\begin{frame}{结论}
    \begin{description}
        \item[I] First of all
        \item[II] Besides
        \item[III] Last but not least
    \end{description}
\end{frame}

\begin{frame}{致谢}
    \begin{block}{致谢}
        谢谢大家。
    \end{block}
\end{frame}

\begin{frame}[allowframebreaks]
    \frametitle{参考文献}
    \printbibliography
\end{frame}

\end{document}
